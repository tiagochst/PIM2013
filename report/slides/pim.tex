% -%-%-%-%-%-%-%-%-%-%-%-%-%-%-%-%-%-%-%-%-%-%-%-%-%
% PIM380                                           %  
% 10/06/2013                                       %
% -%-%-%-%-%-%-%-%-%-%-%-%-%-%-%-%-%-%-%-%-%-%-%-%-%
\documentclass[compress,pdf,11pt,xcolor=dvipsnames]{beamer}

\usepackage[francais]{pim}

\title{Reconstruction 3D temps-réel HD de visages}                               
\usenavigationsymbolstemplate{}
\setbeamertemplate{footline}[] 
%\setbeamertemplate{ffootlineg}{}%remove navigation symbols
%\beamertemplatenavigationsymbolsempty 
%\setbeamertemplate 
%%% title page %%%
\subtitle{PIM380} 
\author[Tiago Siva, Vinicius Gardelli]{
  Tiago Chedraoui Silva \and
  Vinicius Dias de Oliveira Gardelli\\
}
\institute{Télécom Paristech}
\date{Juin 28, 2013}


\begin{document}

\begin{frame}
  \titlepage
\end{frame}

\begin{frame}{Plan}
  \tableofcontents
\end{frame}

\section{Introduction}

\begin{frame}{Les animations faciales}

        \begin{block}{Objectif}
          \begin{itemize}
          \item Capturer des performances expressives très
          détaillées
          \end{itemize}
        \end{block}

        \begin{alertblock}{Préoccupation}
          \begin{itemize}
          \item Temps de configuration 
          \item Invasion au acteur 
          \end{itemize}
        \end{alertblock}

\begin{figure}
  
  \begin{columns}
    \begin{column}{0.5\textwidth}
      \includegraphics[width=\textwidth]{img/Face-Mocap}
    \end{column}
    \begin{column}{0.5\textwidth}
      
      \includegraphics[width=\textwidth]{img/Making-of-Tintin-Spielberg_scaledown_450}
    \end{column}
  \end{columns}
  
\end{figure}

  %  \begin{block}

%    \begin{columns}
%      \begin{column}{0.5\textwidth}
%        \begin{block}{Objectif}
%          Capturer des performances expressives très
%          détaillées
%        \end{block}
%      \end{column}
%      \begin{column}{0.5\textwidth}
%        \begin{alertblock}{Préocupation}
%        Temps de configuration \\
%        Invasive au acteur 
%        \end{alertblock}
%      \end{column}
%      \end{columns}
      
      
%      \begin{description}
%      \item [\color{orange}Objectif \hfill] Capturer des performances
%        expressives très détaillées \hfill 
%      \item [\color{red}{Préocupation}]Temps de configuration,
%        invasive au acteur 
%      \item [\color{blue}{Situation} \hfill] \hfill Plusieurs approches dans les
%        deux décennies
%      \end{description}
%    \end{block}
  \end{frame}

%\subsection{Le context}
\subsection{État de l'art}

\begin{frame}{Les méthodes de capture}
  
        \begin{exampleblock}{Situation}
          \begin{itemize}
          \item Convergence aux captures passives
          \end{itemize}
        \end{exampleblock}
        
        %   \begin{column}{0.5\textwidth}
        
        \begin{figure}
          \centering
          \includegraphics[width=7cm]{img/man}
        \end{figure}

 %   \end{column}
 % \end{columns}
\end{frame}


\section{Projet}

\begin{frame}{Le projet}

  \begin{block}{Disney}
    \begin{itemize}
    \item Capturer des performances expressives très
      détaillées
    \item 16 caméras HD
    \item Studio propre
    \end{itemize}
  \end{block}
  
%  \begin{bkblock}{Matériel}
%    \begin{centering}
%      \includegraphics[scale=0.2]{img/kinect}
%    \end{centering}
%  \end{bkblock}

%  \begin{greyblock}{Environnement}
%    \begin{center}
%      \includegraphics[scale=0.1]{img/linux}
%      \hspace{3mm}
%      \includegraphics[scale=0.3]{img/git}
%      \hspace{3mm}
%      \includegraphics[scale=0.1]{img/openmp}
%      \hspace{3mm}
%      \includegraphics[scale=0.1]{img/qt}
%      \hspace{3mm}
%      \includegraphics[scale=0.3]{img/eigen}
%    \end{center}
%  \end{greyblock}
    
\end{frame}

\begin{frame}{Les étapes}
\begin{figure}[ht!]
  \begin{center}
    \includegraphics[width=\textwidth]{img/disney}
  \end{center}
\end{figure}
\end{frame}

\begin{frame}{}
  \begin{alertblock}{Notre approche:}
    \begin{itemize}
    \item Utiliser une Kinect: une maillage à temps réel
    \end{itemize}
  \end{alertblock}
\end{frame}


\begin{frame}{Les étapes}
\begin{figure}[ht!]
  \begin{center}
    \includegraphics[width=\textwidth]{img/projDiagram}
  \end{center}
\end{figure}
\end{frame}


\begin{frame}{De la capture à la reconstruction finale}
\begin{figure}[ht!]
    \includegraphics[width=\textwidth]{img/projSystem}
\end{figure}
\end{frame}

\section{Résultats}

\begin{frame}{Résultats}
\begin{figure}[ht!]
        \centering
        \begin{subfigure}[b]{0.5\textwidth}
                \centering
                \includegraphics[width=\textwidth]{img/f9}
                \caption{Trame 9}
                \label{fig:trame9}
        \end{subfigure}%
        ~ %add desired spacing between images, e. g. ~, \quad, \qquad etc.
          %(or a blank line to force the subfigure onto a new line)
        \begin{subfigure}[b]{0.5\textwidth}
                \centering
               \includegraphics[width=\textwidth]{img/f10}
                \caption{Trame 10}
                \label{fig:trame10}
        \end{subfigure}
        \caption{Séquence d'ouverture de la bouche}
        \label{fig:trames910}
\end{figure}
\end{frame}{}


\begin{frame}{}
\begin{figure}[ht!]
        \centering
        \begin{subfigure}[b]{0.60\textwidth}
                \centering
                \includegraphics[width=\textwidth]{img/f9disp}
        \end{subfigure}%
        ~ %add desired spacing between images, e. g. ~, \quad, \qquad etc.
          %(or a blank line to force the subfigure onto a new line)
        \begin{subfigure}[b]{0.40\textwidth}
                \centering
                \includegraphics[scale=0.43]{img/ColorChart}
        \end{subfigure}
        \caption{Carte de déplacement entre trames 9 et 10}

\end{figure}
\end{frame}{}

\begin{frame}{}
\begin{figure}[ht!]
  \begin{center}
    \includegraphics[width=\textwidth]{img/f9_plain}
  \end{center}
\end{figure}


\end{frame}{}


\section{Conclusion}

\begin{frame}{Conclusion}
  
  \begin{columns}
    \begin{column}{0.5\textwidth}
     
      \begin{beamerboxesrounded}[shadow=true]{Kinect}
        \begin{itemize}
        \item Faible précision
        \item Carte de profondeur incomplète
        \item Adaptation de l'algorithme
        \item Résultat bruité
        \end{itemize}
      \end{beamerboxesrounded}
    \end{column}    
    
    \begin{column}{0.5\textwidth}
      \begin{exampleblock}{Algorithme}
        \begin{itemize}
        \item Rafinnement récupère grande partie des informations perdues
        \end{itemize}
      \end{exampleblock}

    \end{column}    
  \end{columns}    
  
\end{frame}


\begin{frame}{}
  \begin{figure}
    \begin{centering}
      \Huge Démonstration!
      \par\end{centering}
  \end{figure}
\end{frame}

% 
% Questions
% 
\begin{frame}{}
  \begin{figure}
    \begin{centering}
      \includegraphics[scale=0.1]{img/Icon-round-Question_mark.jpg}
      \par\end{centering}
  \end{figure}
\end{frame}

\end{document}


